% Options for packages loaded elsewhere
\PassOptionsToPackage{unicode}{hyperref}
\PassOptionsToPackage{hyphens}{url}
%
\documentclass[
]{article}
\usepackage{amsmath,amssymb}
\usepackage{lmodern}
\usepackage{iftex}
\ifPDFTeX
  \usepackage[T1]{fontenc}
  \usepackage[utf8]{inputenc}
  \usepackage{textcomp} % provide euro and other symbols
\else % if luatex or xetex
  \usepackage{unicode-math}
  \defaultfontfeatures{Scale=MatchLowercase}
  \defaultfontfeatures[\rmfamily]{Ligatures=TeX,Scale=1}
\fi
% Use upquote if available, for straight quotes in verbatim environments
\IfFileExists{upquote.sty}{\usepackage{upquote}}{}
\IfFileExists{microtype.sty}{% use microtype if available
  \usepackage[]{microtype}
  \UseMicrotypeSet[protrusion]{basicmath} % disable protrusion for tt fonts
}{}
\makeatletter
\@ifundefined{KOMAClassName}{% if non-KOMA class
  \IfFileExists{parskip.sty}{%
    \usepackage{parskip}
  }{% else
    \setlength{\parindent}{0pt}
    \setlength{\parskip}{6pt plus 2pt minus 1pt}}
}{% if KOMA class
  \KOMAoptions{parskip=half}}
\makeatother
\usepackage{xcolor}
\IfFileExists{xurl.sty}{\usepackage{xurl}}{} % add URL line breaks if available
\IfFileExists{bookmark.sty}{\usepackage{bookmark}}{\usepackage{hyperref}}
\hypersetup{
  hidelinks,
  pdfcreator={LaTeX via pandoc}}
\urlstyle{same} % disable monospaced font for URLs
\setlength{\emergencystretch}{3em} % prevent overfull lines
\providecommand{\tightlist}{%
  \setlength{\itemsep}{0pt}\setlength{\parskip}{0pt}}
\setcounter{secnumdepth}{-\maxdimen} % remove section numbering
\ifLuaTeX
  \usepackage{selnolig}  % disable illegal ligatures
\fi

\author{}
\date{}

\begin{document}

\textbf{Élection d'un leader}

On considère un réseau constitué de \emph{n} sites 0,. . .,\emph{n −} 1.
Chaque site \emph{i} a une valeur \emph{vali} avec les \emph{vali} deux
à deux distincts. Le leader est le site avec la plus grande valeur
\emph{vali}. A un moment donné, l'un des sites veut initier une
élection, c'est à dire initier un algorithme distribué qui a pour but de
déterminer lequel des sites est le leader.

\textbf{Exercice : élection dans un anneau unidirectionnel.}

Le réseau est organisé en cycle unidirectionnel : les communications
élémentaires entre sites voisins sont des communications allant de
\emph{i} à \emph{i} + 1 (modulo \emph{n}).\\
On propose l'algorithme suivant pour le site \emph{i}.

\begin{quote}
Initialisation de l'élection ou 1ère réception de message \emph{→}

1. envoyer le message \emph{\textless{}} 1\emph{, vali \textgreater{}} à
\emph{succi}\\
2. \emph{etati ← actif} ;

3. \emph{maxi ← vali} ;

Réception du message \textless1,V\textgreater{} \emph{→}

1. si (\emph{etati} == \emph{actif}) alors

(a) si (\emph{V} ! = \emph{maxi}) alors
\emph{envoyer}(\emph{\textless{}} 2\emph{, V \textgreater{}});
\emph{voisini} = \emph{V}

(b) sinon \emph{maxi} est élu, prevenir les autres,

2. Sinon envoyer le message \emph{\textless{}} 1\emph{, V
\textgreater{}} à \emph{succi}

Réception du message \emph{\textless{}} 2\emph{, V \textgreater→}

1. si (\emph{etati}==actif) alors

(a) si \emph{voisini \textgreater{} V} et \emph{voisin\textgreater maxi}
alors

i. \emph{maxi ← voisini}\\
ii. envoyer \emph{\textless{}} 1\emph{, maxi \textgreater{}} à
\emph{succi}

(b) Sinon \emph{etati ← passif} ;

2. sinon envoyer (\textless2,V\textgreater) à \emph{succi}
\end{quote}

\textbf{Question 1 :} Appliquer cet algorithme sur un exemple sur un
anneau d'au moins 4 sites sachant qu'un seul processeur initie
l'élection.

\textbf{Question 2 :} A quoi servent les variables locales \emph{maxi},
\emph{etati}, \emph{voisini} du site \emph{i} ? A quoi servent les deux
messages ?

\textbf{Question 3 :} Par qui peut être proclamé le résultat ?

1

\textbf{Question 4 :} Quelle est la complexité de l'algorithme en nombre
de messages (au pire des cas) lors que tous les sites commencent
l'élection en même temps ?

\textbf{Exercice : élection dans un anneau bidirectionnel.}

Le réseau est organisé en cycle nidirectionnel : les communications
élémentaires entre sites voisins sont des communications allant de
\emph{i} à \emph{i} + 1 et allant de \emph{i} à \emph{i −} 1 (modulo
\emph{n}).

\textbf{Algorithme 1}

L'algorithme proposé marche par phase et est basé sur le filtrage. Les
sites peuvent être dans deux états perdant ou actif. Lorsqu'un site
reçoit un message contenant une valeur plus grande, il devient perdant.

A chaque phase \emph{j}, les sites encore actifs envoient leur valeur à
une distance de 2\emph{j}d'eux. Lorsqu'un site actif reçoit un tel
message, il devient perdant si la valeur reçue est supérieure à la
sienne sinon il passe à la phase suivante.

Nous allons supposer ici que le réseau est synchrone par phase.
C'est-à-dire que les actifs au début de la phase \emph{j} commencent
tous en même temps à exécuter les opérations de cette phase.

\begin{quote}
1. Donner un exemple d'exécution sur un anneau de 8 sommets.

2. Après la phase \emph{j}, quelle est la distance minimale entre les
deux sites actifs ?

3. Combien y a-t-il de phases ?

4. Donner le nombre de messages échangés.

5. Qui connaît le gagnant ?

6. Donner l'algorithme.

7. Quel est l'inconvénient sur le nombre de messages
\end{quote}

\textbf{Algorithme 2}\\
Voici un algorithme d'élection :

Lors du r\{\textbackslash'e\}veil du site i\\
\{\textbackslash'e\}tat(i) :=actif ;\\
tant que (\{\textbackslash'e\}tat(i) ==actif) faire\\
envoyer(val(i)) \{\textbackslash`a\} gauche\\
envoyer(val(i)) \{\textbackslash`a\} droite\\
Attendre un message Val(g) venant de gauche ; Attendre un message Val(d)
venant de droite ; V:=max\{Val(g), Val(d)\};\\
si ( V \textgreater= val(i)) alors \{\textbackslash'e\}tat(i) :=passif ;
si ( V == val(i)) alors\\
gagnant(i)=i ;\\
Pr\{\textbackslash'e\}venir les autres sites.

\begin{quote}
sinon faire suivre tous les messages.

1. En quoi cet algorithme est-il différent du précèdent ? Peut-on
remplacer \emph{≥} par \emph{\textgreater{}} ?

2. En déduire le nombre de messages échangés pendant l'algorithme.
\end{quote}


\end{document}
