\documentclass{article}
\usepackage{amsmath}
\usepackage{color,pxfonts,fix-cm}
\usepackage{latexsym}
\usepackage[mathletters]{ucs}
\DeclareUnicodeCharacter{46}{\textperiodcentered}
\DeclareUnicodeCharacter{8594}{$\rightarrow$}
\DeclareUnicodeCharacter{58}{$\colon$}
\DeclareUnicodeCharacter{60}{\textless}
\DeclareUnicodeCharacter{62}{\textgreater}
\DeclareUnicodeCharacter{8592}{$\leftarrow$}
\DeclareUnicodeCharacter{8805}{$\geq$}
\usepackage[T1]{fontenc}
\usepackage[utf8x]{inputenc}
\usepackage{pict2e}
\usepackage{wasysym}
\usepackage[english]{babel}
\usepackage{tikz}
\pagestyle{empty}
\usepackage[margin=1in,paperwidth=595pt,paperheight=841pt]{geometry}

\usepackage{enumitem}


\title{Election d'un leader}


\begin{document}

\maketitle


On considère un réseau constitué de $n$ sites $0,\ldots,n - 1$. Chaque site $i$ a une valeur $val_i$ avec $val_i$ deux à deux distincts. Le leader est le site avec la plus grande valeur $val_i$. A un moment donné, l'un des sites veut initier une élection c'est-à-dire initier un algorithme distribué qui a pour but de déterminer lequel des sites est le leader.


\section*{Élection dans un anneau unidirectionnel}
Le réseau est organisé en cycle unidirectionnel : les communications élémentaires entre sites voisins sont des communications allant de $i$ à $i + 1$ (modulo $n$).\\
On propose l'algorithme suivant pour le site $i$.\\

\begin{itemize}[noitemsep]
    \item[] Initialisation de l'élection ou $1^{ere}$ réception de message $\rightarrow$
    \begin{enumerate}[noitemsep]
        \item envoyer le message $<1, val_i>$ à $succ_i$
        \item $etat_i$ $\leftarrow$ $actif$
        \item $max_i$ $\leftarrow$ $val_i$
    \end{enumerate}
    \item[] Réception du message $<1, V>$ $\rightarrow$
    \begin{enumerate}[noitemsep]
        \item si ($etat_i == actif$) alors
        \begin{enumerate}
            \item si ($V != max_i)$ alors $envoyer(<2, V>); voisin_i = V$
            \item sinon $max_i$ est élu, prévenir les autres
        \end{enumerate}
        \item Sinon envoyer le message $<1, V>$ à $succ_i$
    \end{enumerate}
    \item[] Réception du message $<2, V>$ $\rightarrow$
    \begin{enumerate}[noitemsep]
        \item si ($etat_i == actif$) alors
        \begin{enumerate}
            \item si $voisin_i > V$ et $voisin > max_i$ alors
            \begin{enumerate}
                \item $max_i \leftarrow voisin_i$
                \item envoyer < 1, $max_i$ > à $succ_i$
            \end{enumerate}
            \item Sinon $etat_i \leftarrow passif$
        \end{enumerate}
        \item sinon envoyer $(<2, V>)$ à $succ_i$
    \end{enumerate}
\end{itemize}

\subsection*{Question 1:} Appliquer cet algorithme sur un exemple sur un anneau d'au moins 4 sites sachant qu'un seul processeur initie l'élection.

\subsection*{Question 2:} A quoi servent les variables locales $max_i$, $etat_i$, $voisin_i$ du site $i$ ? A quoi servent les deux messages ?

\subsection*{Question 3:} Par qui peut-être proclamé le résultat ?

\subsection*{Question 4:} Quelle est la complexité de l'algorithme en nombre de messages (au pire des cas) lors que tous les sites commencent l'élection en même temps ?


\section*{Election dans un anneau bidirectionnel.}

Le réseau est organisé en cycle nidirectionnel : les communications
élémentaires entre sites voisins sont des communications allant de
\emph{i} à \emph{i} + 1 et allant de \emph{i} à \emph{i −} 1 (modulo
\emph{n}).

\subsection*{Algorithme 1}

L'algorithme proposé marche par phase et est basé sur le filtrage. Les
sites peuvent être dans deux états perdant ou actif. Lorsqu'un site
reçoit un message contenant une valeur plus grande, il devient perdant.

A chaque phase \emph{j}, les sites encore actifs envoient leur valeur à
une distance de 2\emph{j}d'eux. Lorsqu'un site actif reçoit un tel
message, il devient perdant si la valeur reçue est supérieure à la
sienne sinon il passe à la phase suivante.

Nous allons supposer ici que le réseau est synchrone par phase.
C'est-à-dire que les actifs au début de la phase \emph{j} commencent
tous en même temps à exécuter les opérations de cette phase.

\end{document}
