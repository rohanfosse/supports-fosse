\documentclass[10pt,a4paper]{article}
\usepackage[utf8x]{inputenc}
\usepackage{ucs}
\usepackage{amsmath}
\usepackage{amsfonts}
\usepackage{amssymb}
\usepackage{a4wide}
\usepackage{comment}
\usepackage[pdftex]{graphicx}
\usepackage{epstopdf}
\newcommand{\cmd}[1]{\texttt{#1}}
\newcommand{\remove}{}
\newcommand{\dir}[1]{\textsf{#1}}
\newcommand{\pop}{Depiler\xspace}
\newcommand{\push}{Empiler\xspace}
\newcommand{\tete}{LireTete}
\newcommand{\sommet}{LireSommet}
\newcommand{\empt}{EstVide}
\newcommand{\enqueue}{Enfiler\xspace}
\newcommand{\dequeue}{Defiler\xspace}
\newcommand{\get}{\ensuremath{\leftarrow\ }}


\usepackage{comment}
\usepackage{algorithm}
\usepackage[noend]{algpseudocode}
\usepackage{tcolorbox}

\newtcolorbox{mybox}[3][]
{
  colframe = #2!25,
  colback  = #2!10,
  coltitle = #2!20!black,  
  title    = {#3},
  #1,
}
\newcommand*\Let[2]{\State #1 $\gets$ #2}


\excludecomment{solution}
\includecomment{solution}

\title{IF111 - Algorithmes et structures de données-TD2 :\\ Piles, Files et Listes}
\date{}
\author{\underline{Jonathan Narboni}, Rohan Fossé}
\date{\underline{\texttt{jonathan.narboni@labri.fr}}, \texttt{rfosse@labri.fr}}


\begin{document}
\maketitle

\section*{Exercice 1}
On dispose d'une pile contenant des nombres entiers. Écrire un algorithme qui permet d'afficher le contenu de la pile. On veut voir s'afficher dans l'ordre le sommet, le second élément, et ainsi de suite. On prendra soin que la pile, à l'issue de l'affichage, contienne les mêmes éléments \emph{dans le même ordre}. L'algorithme utilise une pile auxiliaire. Comment modifier cet algorithme pour afficher les éléments dans l'ordre inverse ?


\begin{tcolorbox}
\begin{tabbing}
	~~~~\=~~~~\=~~~~\=~~~~\=~~~~\=\kill
	\>  \textbf{AffichePile(P)}\\
	 1.\> \{\\
	 2.\> \> \textbf{entier} $x$;\\
         3.\> \> $P_{aux} \leftarrow CreerPileVide()$;\\
         4.\> \> \textbf{tant que} ! $ PileVide(P)$\\
         5.\> \>  $x\leftarrow Depiler(P)$\\
         6.\> \>  $afficher(x)$;\\
         7.\> \>  $Empiler(P_{aux},x)$;\\
         8.\> \> \textbf{fin tant que};\\
         9.\> \> \textbf{tant que} \ $! PileVide(P_{aux})$\\
        10.\> \> $Empiler(P,Depiler(P_{aux}))$;\\
        11.\> \> \textbf{fin tant que}\\
	12.\>  \}
\end{tabbing}

Pour afficher en ordre inverse il suffit de afficher à la ligne 10 quand on depiler de $P_{aux}$ et avant qu'on empile en P.
\end{tcolorbox}

\section*{Exercice 2}
On veut écrire des algorithmes pour manipuler des files et des piles. Les algorithmes peuvent utiliser des piles et des files auxiliaires.  
\begin{itemize}
\item On se donne une pile $P1$ contenant des entiers positifs. Ecrire un algorithme pour déplacer les entiers de $P1$ dans une pile $P2$ de façon à avoir dans $P2$ tous les nombres pairs en dessous des nombres impairs. L'algorithme prend en entrée $P1$ et $P2$ et utilise $O(1)$ mémoire auxiliaire.

\begin{tcolorbox}
\begin{tabbing}
	~~~~\=~~~~\=~~~~\=~~~~\=~~~~\=\kill
	\>  \textbf{Deplace($P_1$,$P_2$)}\\
	 \> \{\\   \textbf{tant que} ! \texttt{EstVide($P_1$)} faire\\
               \>         $x  \leftarrow$ \texttt{Depiler($P_1$)}\\
        \>  \textbf{si} x est pair alors\\
             \textbf{tant que} \texttt{Sommet($P_2$)} est impaire faire\\
                \texttt{Empiler}($P_1$,\texttt{Depiler}($P_2$))\\
           \textbf{fin tant que}\\
        \textbf{fin si}\\
        \texttt{Emplier$(P_2$, x)}\\
   \textbf{fin tant que}\\

              \> \> \> \textbf{fin tant que}\\
        \> \} 
\end{tabbing}
\end{tcolorbox}



\item Ecrire un algorithme qui renverse une file. Faites de m\^eme pour une pile.

\begin{tcolorbox}
\begin{tabbing}
	~~~~\=~~~~\=~~~~\=~~~~\=~~~~\=\kill
	\>  \textbf{RenverseFile(F)}\\
	 1.\> \{\\
	 2.\> \> \textbf{entier} $x$;\\
         3.\> \> $P_{aux} \leftarrow CreerPileVide()$;\\
         4.\> \> \textbf{tant que} $! FileVide(F)$\\
         5.\> \>  $x\leftarrow Defiler(F)$\\
         6.\> \> $Empiler(P_{aux},x)$;\\
         7.\> \> \textbf{fin tant que};\\ 
         8.\> \> \textbf{tant que} $! PileVide(P_{aux})$\\
         9.\> \> $Enfiler(F,Depiler(P_{aux}))$;\\
        10.\> \> \textbf{fin tant que}\\
        11.\> \} 
\end{tabbing}

\begin{tabbing}
	~~~~\=~~~~\=~~~~\=~~~~\=~~~~\=\kill
	\>  \textbf{RenversePile(P)}\\
	 1.\> \{\\
	 2.\> \> \textbf{entier} $x$;\\
         3.\> \> $F_{aux} \leftarrow CreerFileVide()$;\\
         4.\> \> \textbf{tant que} $!PileVide(P)$\\
         5.\> \>  $x \leftarrow Depiler(P)$\\
         6.\> \> $Enfiler(F_{aux},x)$;\\
         7.\> \> \textbf{fin tant que};\\ 
         8.\> \> \textbf{tant que} $! FileVide(F_{aux})$\\
         9.\> \> $Empiler(P,Defiler(F_{aux}))$;\\
        10.\> \> \textbf{fin tant que}
        11.\> \} 
\end{tabbing}
\end{tcolorbox}




\item En utilisant seulement des files, écrire un algorithme qui prend en entrée une file $F$ contenant les éléments triés en ordre croissant (la tête est le plus petit élément de la file) et un élément $e$ à enfiler. L'algorithme retournera la file $F$ contenant l'élément $e$ tout en préservant l'ordre croissant des éléments. 

\begin{tcolorbox} 
\begin{tabbing}
	~~~~\=~~~~\=~~~~\=~~~~\=~~~~\=\kill
	\>  \textbf{TriMain(F,e)}\\
	 \> \{\textbf{entier} $x$;\\
	 \> \ \textbf{bool} $insere\leftarrow faux$;\\
        \> \ $F_{a} \leftarrow CreerFileVide()$;\\
         \> \ \textbf{tant que} $(EstVide(F)=faux)$ \textbf{et} $(insere=faux)$ faire\\
           \> \> $x\leftarrow Defiler(F)$;\\
          \> \> \textbf{si} $x<e$ \textbf{alors} $Enfiler(F_{a},x)$;\\
         \> \> \textbf{sinon} \\
           \> \> \ $Enfiler(F_{a},e)$; \\
           \> \> \ $Enfiler(F_{a},x)$;\\
           \> \> \ $insere\leftarrow vrai$;\\
           \> \> \textbf{fin si}\\ 
 \> \textbf{fin tant que} \\
          \> \textbf{si} $(insere=faux)$ \textbf{alors} $Enfiler(F_{a},e)$;\\   % F est vide on insere e en dernier
          \> \textbf{sinon} \\
           \> \> \textbf{tant que} $(EstVide(F)=faux)$\\ % on est sortie de la boucle parce qu'on a insere e
         \> \> $Enfiler(F_{a},Defiler(F))$;\\
        \> \> \textbf{fin tant que}\\
\> \textbf{fin si}\\ 
 \> \textbf{tant que} $(EstVide(F_{a})=faux)$ \\
         \> $Enfiler(F,Defiler(F_{a}))$;\\
        \> \textbf{fin tant que}\\
        \> \} 
\end{tabbing}
\end{tcolorbox}


\end{itemize}

\section*{Exercice 3}
On veut implémenter une file en utilisant deux piles. Écrire les opérations suivantes : \texttt{EstVide(F)}, 
retourne vrai si la file est vide, faux sinon; \texttt{Enfiler(F,e)} qui insère $e$ en queue de la file $F$, \texttt{Defiler(F)}
retourne l’ élément de tête et le supprime de la file, \texttt{Tête(F)} qui retourne l’ élément de tête sans le supprimer de la file.
Pour toute pile $P$ vous avez à disposition les opérations de base suivantes pour la manipuler : \texttt{EstVide(P)},  \texttt{Empiler(P,e)}, \texttt{Depiler(P)} et \texttt{Sommet(P)}.

\begin{tcolorbox} 
\textbf{Une solution :} Nous utiliserons deux piles $P_1$ et $P_2$ pour implémenter le type abstrait file. L'objet file aura deux attributs de type pile appelés $P_1$ et $P_2$.
L'idée est que la tête de la file $F$ correspond au sommet de la pile $P_1$ et le dernier élément de la file est le sommet de la pile $P_2$. La file est vide si les deux piles $P_1$ et $P_2$ sont vides. 
\\ \textbf{Enfiler($F,e$)}: Enfile un élément $e$ dans la file $F$. Ça correspond à empiler l'élément dans la pile $P_2$. \\ 
\textbf{Defiler($F$)} retourne la tête de la file, c'est-à-dire le premier élément inséré dans $F$. Cette opération est implémentée de la manière suivante : si la file $P_1$ n'est pas vide on retourne l'élément dépilé de $P_1$. Si $P_1$ est vide mais $P_2$ ne l'est pas, on reverse tous les éléments de $P_2$ dans $P_1$ et on dépile le sommet de $P_1$. \\
\texttt{Tête(F)} est similaire à \textbf{Defiler($F$)}, mais on utilisera l'opération \texttt{Sommet($P_1$)} à la place de \texttt{Depiler}($P_1$).
 
\begin{tabbing}
	~~~~\=~~~~\=~~~~\=~~~~\=~~~~\=\kill
	\textbf{EstVide($F$)}\\
	 \{ \textbf{retourner} \texttt{EstVide($P_1$)} ET \texttt{EstVide($P_2$)}\} 
        \\
     \\
       \textbf{Enfiler($F,e$)}\\
 \{\texttt{Empiler($P_2,e$)} \} \\
	 \\ 
	
  \textbf{Defiler($F$)}\\
    \{ \textbf{si} \texttt{EstVide($P_1$)}=Faux \textbf{alors} \textbf{retourner} \texttt{Depiler}($P_1$);\\
       \ \ \textbf{sinon} \\
           \> \textbf{si} \texttt{EstVide($P_2$)}=Faux \textbf{alors} \\
        \> \> \textbf{tant que} $\texttt{EstVide}(P_2)=Faux$ \textbf{faire}\\
       \> \> \>  \texttt{Empiler}($P_1$,\texttt{Depiler}($P_2$));\\
        \> \> \textbf{fin tant que}\\
         \> \>  \textbf{retourner} \texttt{Depiler}($P_1$);\\
          \> \textbf{fin si}\\
           \> \textbf{retourner} ``ERREUR file vide'' \\
          \ \textbf{fin si} \\
        \} 
\end{tabbing}
\end{tcolorbox}


\section*{Exercice 4}
On veut manipuler des listes. Écrivez les fonctions suivantes de manière récursive en utilisant les primitives \texttt{tête(L)}, \texttt{queue(L)} et \texttt{estVide(L)}. %\texttt{creerListeVide()},  et \texttt{ajouterEnTête(a,L)}.
(on suppose que toutes les listes contiennent des entiers) 
\begin{enumerate}
 \item paire(L) qui retourne $vrai$ si la liste L ne contient que des entiers pairs, $faux$ sinon.
 
 
 \begin{tcolorbox} 
 
\begin{tabbing}
	~~~~\=~~~~\=~~~~\=~~~~\=~~~~\=\kill
	\textbf{paire(L)}\\
    \{ \textbf{si} \texttt{EstVide(L)}  \textbf{alors} \textbf{retourner} $vrai$; \textbf{fin si} \\
     \textbf{si} (\texttt{tête(L)}$\%$ 2=0) \textbf{alors} \textbf{retourner} \textbf{paire(\texttt{queue(L)})} \\
       \ \textbf{sinon} \texttt{retourner} $faux$\\
       \ \textbf{fin si}\\
       \} 
\end{tabbing}
\end{tcolorbox}
 
%\item additionner(x, L) qui ajoute x à chaque élément de la liste L. 


\item avant$\_$dernier(L) qui renvoie l'avant-dernier élément de la liste L s'il existe, $faux$ sinon.

\begin{tcolorbox} 
 
\begin{tabbing}
	~~~~\=~~~~\=~~~~\=~~~~\=~~~~\=\kill
	\textbf{avant$\_$dernier(L)}\\
    \{ \textbf{si} \texttt{EstVide(L)}  \textbf{alors} \textbf{retourner} $faux$; \textbf{fin si}\\
      \ \textbf{si} \texttt{EstVide(queue(L))}  \textbf{alors} \textbf{retourner} $faux$; \textbf{fin si}\\
      \  \textbf{si} \texttt{EstVide(queue(queue(L)))} \textbf{alors} \textbf{retourner}(\texttt{tête(L)}) \\
      \      \textbf{sinon} \texttt{retourner} \textbf{avant$\_$dernier(queue(L))}\\
      \      \textbf{fin si}\\
       \} 
\end{tabbing}
\end{tcolorbox}



%\item apparait(x, n, L) qui retourne $vrai$ si x apparait au moins $n\geq 0$ fois dans L, $faux$ sinon.

%\begin{tcolorbox} 
 
%\begin{tabbing}
%	~~~~\=~~~~\=~~~~\=~~~~\=~~~~\=\kill
%	\textbf{apparait(x, n, L)}\\
%    \{ \textbf{si} n=0  \textbf{alors} \textbf{retourner} $vrai$; \textbf{fin si}\\
%      \ \textbf{si} \texttt{EstVide(L)}  \textbf{alors} \textbf{retourner} $faux$; \textbf{fin si}\\
%      \ \textbf{si} \texttt{Tête(L)}=x \textbf{alors} \textbf{retourner} \textbf{apparait(x, n-1, queue(L))} \\
%      \  \textbf{sinon}  \textbf{retourner} \textbf{apparait(x, n, queue(L))}\\
%      \  \textbf{fin si}\\
%       \} 
%\end{tabbing}
%\end{tcolorbox}


%\item alterne(L) qui teste si L est alterné,  c'est-à-dire si son premier élément est plus petit que le second, que le second est plus grand que le troisième, le troisième est plus petit que le quatrième, le quatrième plus grand que le cinquième et ainsi de suite.
\end{enumerate}
\end{document}
